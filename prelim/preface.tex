\chapter*{Preface}      % starred form omits the `chapter x'
\addcontentsline{toc}{chapter}{Preface}

This thesis focuses primarily on searches for pair production of Higgs bosons in the \bbbb final state. It begins 
with an overview of the relevant physics and experimental background for such work, structured as follows:
In Chapter \ref{chap:intro-SM}, I provide an overview of the Standard Model of particle physics, with discussion 
of the theoretical and experimental development of such a model. Chapter \ref{chap:hh-bsm} dives more into the 
details of Higgs boson pair production, as well as the physics beyond the Standard Model relevant for 
this thesis. Chapter \ref{chap:experiment} then provides an introduction to the experimental apparatus used for 
the presented searches, with an outline of the Large Hadron Collider and the ATLAS detector. Chapter \ref{chap:simulation} 
details the procedure to simulate the physics processes discussed in Chapters \ref{chap:intro-SM} and \ref{chap:hh-bsm}, 
including simulation of the detector discussed Chapter \ref{chap:experiment}. Finally, a review of the procedures to 
reconstruct objects used for physics analysis is provided in Chapter \ref{chap:reconstruction}, with a focus on jets and flavor-tagging.

The original contributions of this thesis are discussed in a variety of places. Chapter \ref{chap:simulation} 
includes my work on the development of methods to improve the modeling 
of hadronic showers within a parametrized simulation of the ATLAS calorimeter. I entirely developed both the method and 
the software for the Gaussian method discussed in Chapter \ref{chap:simulation}, including all of the validations 
presented there. The development of the Variational Autoencoder method was done in collaboration with Dalila Salamani. 
This work has been published in a set of proceedings~\cite{CHEP-proceedings} and implemented into ATLAS software. At the 
time of this writing, it is a candidate for inclusion in the Run 3 simulation infrastructure.

Chapters \ref{chap:bbbb-intro} through \ref{chap:bbbb-results}  detail searches for resonant 
and non-resonant pair production of Higgs bosons in the \bbbb final state. I was one of the main analyzers for both of 
these searches, responsible for much of the development of the methods, infrastructure, and documentation. 
My most major contribution was the development of the background estimation procedure and the associated uncertainties, 
which I spearheaded both conceptually and practically. This is quite a significant contribution for both the resonant and non-resonant, as it is the core of much of the analysis design, with the most direct impact on the final results -- to paraphrase Georges Aad during the resonant review process, ``This is the analysis.''

This was not my only contribution -- for the resonant search, I contributed to the development of the analysis selection 
and codebase, performed many of the necessary cross-checks, and was the co-editor of the ATLAS internal documentation, 
along with Beojan Stanislaus, who developed the BDT pairing and much of the analysis software. Credit goes as well to 
Lucas Borgna, for much of the work behind the development of the trigger strategy.

The resonant search follows many of the procedures of the early Run 2 analysis~\cite{EXOT-2016-31}, 
with the pairing method and background estimation method constituting the two biggest analysis-level differences 
from that work. The non-resonant analysis has several additional changes, which include a variety of new kinematic variable 
and region definitions, as well as a different pairing method than both the early Run 2 search and the resonant search. 
I was responsible for a large majority of the studies behind each of these decisions. I am also responsible for the 
development of much of the modern $4b$ software infrastructure, including, of course, the background estimation framework, 
a new limit setting framework, and a new centralized plotting framework, the latter of which greatly facilitates 
both studies and documentation for the more complicated non-resonant analysis strategy.

At the time of this writing, the preliminary resonant results have been published~\cite{ATLAS-CONF-2021-035}, with a 
paper soon to follow, pending some additional studies on the high mass ($>\SI{3}{\TeV}$) results in the boosted 
analysis channel~\footnote{This thesis focuses on the resolved analysis channel, so these additional studies do 
not impact the final results of this thesis work. The boosted channel is included in the limits presented 
in Figure \ref{fig:res-limits}, but in no other plots or results. See Appendix \ref{app:other-channels} for a description 
of the boosted analysis selection.}. The non-resonant results are more preliminary, but the analysis strategy presented 
in this thesis is approximately final, and the analysis is beginning internal ATLAS review.

While these above results are the main results of this thesis, proof-of-concept studies for two novel $4b$ analysis methods 
are included in Appendix \ref{app:future}. This work, done in collaboration primarily with 
Nicole Hartman, was not used for the $4b$ results presented here, but I encourage the interested reader to consider 
these for further study in future iterations of the $4b$ analysis. I note as well that, while these methods are 
promising in the context of the $4b$ analysis, they are also methodologically interesting, and conceptually 
related results have been published concurrently with the development of the work presented in this thesis in 
\cite{SPANet} and \cite{Astro-NSF}. 
