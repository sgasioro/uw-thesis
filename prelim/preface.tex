\chapter*{Preface}      % starred form omits the `chapter x'
\addcontentsline{toc}{chapter}{Preface}

This thesis focuses primarily on searches for pair production of Higgs bosons in the \bbbb final state. 
In Chapter \ref{chap:intro-SM}, I provide an overview of the Standard Model of particle physics, with discussion 
of the theoretical and experimental development of such a model. Chapter \ref{chap:hh-bsm} dives more into the 
details of Higgs boson pair production, as well as the physics beyond the Standard Model relevant for 
this thesis. Chapter \ref{chap:experiment} then provides an introduction to the experimental apparatus used for 
the presented searches, with an outline of the Large Hadron Collider and the ATLAS detector. 

Chapter \ref{chap:simulation} details the procedure to simulate the physics processes discussed in 
Chapters \ref{chap:intro-SM} and \ref{chap:hh-bsm}, including simulation of the detector discussed Chapter \ref{chap:experiment}. 
A review of the procedures to reconstruct objects used for physics analysis is provided in 
Chapter \ref{chap:reconstruction}, with a focus on jets and flavor-tagging. To conclude the introductory material of 
the thesis, a discussion of the general procedures behind a physics search at the LHC is provided in 
Chapter \ref{chap:search-outline}.

While this thesis provides a review of the necessary background information, it also presents a significant body of 
original research. Chapter \ref{chap:simulation} includes my work on the development of methods to improve the modeling 
of hadronic showers within a parametrized simulation of the ATLAS calorimeter. I entirely developed both the method and 
the software for the Gaussian method discussed in Chapter \ref{chap:simulation}, including all of the validations 
presented there. The development of the Variational Autoencoder method was done in conjuction with Dalila Salamani, but 
also contains significant contribution from me.

The content presented in Chapters \ref{chap:bbbb} and \ref{chap:future} is almost entirely original research. 
Chapter \ref{chap:bbbb} details searches for resonant and non-resonant pair production of Higgs bosons in the \bbbb 
final state. I was one of the main analyzers for both of these searches, but my dominant contribution was the 
development of the background estimation procedure and the associated uncertainties, which I spearheaded both conceptually 
and practically. This is quite a significant contribution for both the resonant and non-resonant -- to paraphrase Georges 
Aad during the resonant review process, ``This is the analysis.''

This was not my only contribution -- for the resonant search, I contributed to the development of the analysis selection 
and limit setting, as well as many many cross checks, and was the co-editor of the ATLAS internal documentation, along with 
Beojan Stanislaus, who developed the BDT pairing and much of the analysis software. Credit goes as well to Lucas Borgna, 
for much of the work behind the development of the trigger strategy.

The resonant search follows many of the procedures of the early Run 2 analysis~\cite{EXOT-2016-31}, 
with the pairing method and background estimation method 
constituting the two biggest analysis-level differences from that work. However, the non-resonant analysis has 
several additional changes, including various kinematic variable and region definitions 
and a different pairing method than both the early Run 2 search and the resonant search. I was responsible for a 
large majority of the studies behind each of these decisions, literally improving the Standard Model sensitivity 
by a factor of three relative to the resonant analysis strategy baseline. I am also responsible for the development 
of much of the modern $4b$ software infrastructure, 
including, of course, the background estimation framework, a new limit setting framework, and a new centralized plotting 
framework, the latter of which greatly facilitates both studies and documentation for the more complicated 
non-resonant analysis strategy.

Chapter \ref{chap:future} is also entirely original research, done in collaboration primarily with Nicole Hartman, presenting two novel methods for the future of the $4b$ channel. While these represent promising directions 
for the $4b$ analysis, they are also methodologically interesting, and conceptually related results were published 
concurrently with the development of the work presented in this thesis in \cite{SPANet} and \cite{Astro-NSF}.
