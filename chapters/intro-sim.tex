\chapter{Simulation}
\label{chap:simulation}

Simulated physics samples are a core piece of the physics output of the Large Hadron Collider, 
providing a map from a physics theory into what is observed in our detector. This is crucial for 
searches for new physics, where simulation is necessary to describe what a given signal model looks 
like, but also extremely valuable for describing the physics of the Standard Model, providing detailed 
predictions of background processes for use in everything from designing simple cuts to training 
multivariate discriminators. Broadly, simulation can be split into two stages: \emph{event generation}, in which 
physics theory is used to generate a description of particles present after a proton-proton collision, and 
\emph{detector simulation}, which passes this particle description through a simulation of the detector 
material, providing a view of the physics event as it would be seen in ATLAS data. Such simulation 
is often called Monte Carlo in reference to the underlying mathematical framework, which relies on random 
sampling. 

\section{Event Generation}
A variety of tools are used to simulate various aspects of event generation. \MADGRAPH\cite{MG5} 
is commonly used for the generation of the ``hard scatter'' event, i.e., two protons collide and 
some desired physics process happens. In practice, this is not quite as simple as two quarks or 
gluons interacting. Protons are composed of three ``valence'' quarks with various momenta interacting 
with each other via exchange of gluons, but also a sea of virtual gluons which may decay into other
quarks. A hard scatter event is therefore characterized by the corresponding particle level 
diagrams, but additionally by a set of \emph{parton distribution functions} (PDFs), which describe the 
probablity to find constituent quarks or gluons at carrying various momenta at a given energy 
scale (often written $Q^2$). Such PDFs are measured experimentally \todo{cite} and the selection 
of a ``PDF set'' and a given physics process characterizes the hard scatter.

The hard scatter is not the only component of a given collider event, however. Incoming and outgoing 
particles are themselves very energetic and may radiate particles along their trajectory. In particular,
gluons, which have a self-interaction term as described in Chapter \ref{chap:intro-SM}, may be radiated,  
which subesquently themselves radiate gluons or decay to quarks which can also radiate gluons, in a 
whole mess of QCD that both contributes to the particle content of a collider event and is not directly 
described by the hard scatter. This cascade, called a \emph{parton shower}, has a dedicated set of simulation
tools, commonly \HERWIGV{7}~\cite{Herwig7}\cite{HerwigPP} and \PYTHIAV{8}~\cite{Pythia}, which interface with 
tools such as \MADGRAPH for simulation. 

Due to color confinement (Chapter \ref{chap:intro-SM}), quarks and gluons cannot be observed free particles, but 
rather undergo a process called hadronization, in which they are grouped into colorless hadrons (e.g. \emph{mesons},
consisting of one quark and one antiquark). In simulation, this is also handled with \HERWIGV{7} and \PYTHIAV{8}.

The physics of \Pqb-quarks is quite important for a variety of searches for new physics and measurements of 
the Standard Model, including this thesis work \todo{ref flavor tagging sec?}. Correspondingly, the decay of ``heavy flavor'' particles 
(e.g. $B$ and $D$ mesons, containing $b$ and $c$ quarks respectively) has been very well studied, and 
a dedicated simulation tool, \textsc{EvtGen}~\cite{EvtGen}, is used for such processes.

\todo{add nice parton shower image}

\section{Detector Simulation}
\todo{describe showering in detector section}