\chapter{Future Ideas for $HH\rightarrow \bbbb$}
\label{chap:future}

The searches presented in this thesis make use of a large suite of sophisticated techniques, selected 
through careful study and validation. During this process, a variety of interesting 
directions for the $HH\rightarrow b\bar{b}b\bar{b}$ analysis were explored by this thesis author, 
in collaboration with a few others\footnote{Notably Nicole Hartman (SLAC), who spearheaded much of the 
development and proof of concept work, in collaboration with Michael Kagan and Rafael Teixeira De Lima.}, 
but were not used due to a variety of constraints. We present two such interesting directions here, 
with the hope of encouraging further exploration of these techniques in future work.

\section{pairAGraph: A New Method for Jet Pairing}
As discussed in Chapter \ref{chap:bbbb}, one of the main problems to solve is the pairing of 
$b$-jets into Higgs candidates. Figure \ref{fig:pairing-massplanes} demonstrates that the choice 
of the pairing method, while important for achieving good reconstruction of signal events, also 
significantly impacts the structure of non-$HH$ events, leading to various biases in the 
background estimate. Evaluation of the pairing method therefore must take both of these factors 
into account. While we have presented some advantages in respective contexts for the pairing 
methods considered here, we of course would like to explore further improvements to this important 
component of the analysis.

To that end, we note that all of the pairing methods considered here share a common feature: 
four jets are selected, and the pairing is some discrimination between the available three pairings 
of these four jets. For the methods used in this analysis, the jet selection proceeds via a 
simple $p_{T}$ ordering, with $b$-tagged jets recieving a higher priority than non-tagged jets.

With the advent of a variety of machine learning methods for dealing with a variable number of 
inputs \todo{cite: RNNs, Deep Sets, GNNs, Transformers}, a natural place to improve on the 
pairing is to consider more than just four jets. The pairing and jet selection is then performed 
simultaneously, allowing for the incorporation of more event information in the pairing decision and 
the incorporation of jet correlation structure in the jet selection.

In practice, the majority of $HH\rightarrow\bbbb$ events have either four or five jets which pass 
the kinematic preselection, and any gain from this additional freedom would come from events with 
greater than or equal to five jets. However, this five jet topology is particularly exciting 
for scenarios such as events with initial state radiation (ISR), in which the $HH->4b$ jets are offset by 
a single jet with $p_{T}$ similar in magnitude to that of the $HH->4b$ system. Such events have explicit 
event level information which is not encoded with the inclusion of only the $HH->4b$ jets, and are 
pathological if the ISR jet happens to pass $b$-tagging requirements.

Additionally, with the use of lower tagged regions for background estimation and alternate 
signal regions, this extra flexibility in jet selection may provide a very useful bias -- 
since the algorithm is trained on signal, the selected jets for the pairing will be the 
most ``4b-like'' jets available in the considered set.

For the studies considered here, a transformer \todo{cite Attention} based architecture is used. This is 
best visualized by considering the event as a graph with jets corresponding to nodes and 
edges corresponding to potential connections -- for this reason, we term this algorithm ``pairAGraph''.
The approach is as follows: each jet, $i$, is represented by some vector of input variables, 
$\vec{x}_i$, in our case the four-vector information, $(p_{T}, \eta, \phi, E)$ of each jet, plus 
information on the $b$-tagging decision. A multi-layer perceptron (MLP) is used to create a latent 
embedding, $\mathbf{h}(\vec{x}_i)$, of this input vector.

To describe the relationship between various jets in the event, we then define a vector $\vec{z}_{i}$ 
for each jet as 
\begin{equation}
\vec{z}_i = \sum\limits_j w_{ij}\mathbf{h}(\vec{x}_j)
\end{equation}
where $j$ runs over all jets in the event (including $i=j$), the $w_{ij}$ can be thought of as 
edge weights, and $\mathbf{h}(\vec{x}_j)$ is the latent embedding for jet $j$ mentioned above.

Within this formula, both $\mathbf{h}$ and the $w_{ij}$ are learnable. To learn an appropriate 
latent mapping and set of edge weights, we define a similarity metric corresponding to each 
possible jet pairing:
\begin{equation}
\vec{z}_{1a}\cdot \vec{z}_{1b} + \vec{z}_{2a}\cdot \vec{z}_{2b}
\end{equation}
where subscripts $1a$ and $1b$ correspond to the two jets in pair 1, $2a$ and $2b$ to 
the jets in pair 2 for a given pairing of four distinct jets.

This similarity metric is calculated for all possible pairings, which are then passed through a 
softmax \todo{cite} activation function, which compresses these scores to between $0$ and $1$ 
with sum of $1$, lending an interpretation as probability of each pairing.

In training, the ground truth pairing is set by \emph{truth matching} jets to the $b$-jets 
in the $HH$ signal simulation -- a jet is considered to match if it is $< 0.3$ in $\Delta R$
away from a $b$-jet in the simulation record. Given this ground truth, a cross-entropy loss \todo{cite}
is used on the softmax outputs, and $w_{ij}$ and $\mathbf{h}$ are updated correspondingly.
Training in such a way corresponds to updating $w_{ij}$ and $\mathbf{h}$ to maximize the similarity 
metric for the correct pairing.

In evaluation, the pairings with a higher score (and therefore higher softmax output) given 
the trained $h$ and $w_{ij}$ therefore correspond to the pairings that are most ``HH-like''. 
The maximum over these scores is therefore the pairing used as the predicted result from 
the algorithm.

Because the majority of $HH\rightarrow\bbbb$ events have either four or five jets, it was 
found to be sufficient to only consider a maximum of 5 jets. Consideration of more is in 
principle possible, but the quickly expanding combinatorics leads to a rapidly more 
difficult problem. The jets considered are the five leading jets in $p_{T}$. Notably, 
this set of jets may include jets which are not $b$-tagged, even for the nominal $4b$ 
region -- therefore events with $4$ $b$-jets are not required to use all of them 
in the construction of Higgs candidates, in contrast to the other algorithms used in this 
thesis.

\section{Background Estimation with Mass Plane Interpolation}
The choice of a pairing algorithm that results in a smooth mass plane (such as $\min{\Delta R}$) 
opens up a variety of options for the background estimation. While the method based on 
reweighting of $2b$ events used for this thesis performs well and has been extensively 
studied and validated, it also relies on several assumptions. 
