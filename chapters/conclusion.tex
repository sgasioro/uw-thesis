\chapter{Conclusions}
\label{chap:conclusions}

This thesis has provided an overview of the Standard Model, with an emphasis on pair production of Higgs bosons and how this process may be used to both verify the Standard Model and to search for new physics. An overview of the Large Hadron 
Collider and the ATLAS detector has been provided, and the design and use of simulation infrastructure has been explained, 
including work to improve hadronic shower modeling in fast detector simulation. The translation of detector level 
information to analysis level information has been explained, with an emphasis on jets and the identification of $B$ hadron 
decay. Finally, two searches for Higgs boson pair production have been presented, with a complete set of results for 
resonant production included, focusing on searches beyond the Standard Model, and a preliminary set of results for 
non-resonant production, targeting Standard Model production, with variations of the Higgs self-coupling.

This thesis represents a powerful contribution in multiple areas. My simulation work provides two powerful options for the 
improvement of hadronic showering in fast calorimeter simulation, and is emblematic of a hybrid approach to fast simulation 
which may leverage both expert knowledge and the power of machine learning methods. The $HH$ results presented in this 
thesis are leading contributions to the full Run 2 statement from ATLAS about Higgs boson pair production, and improve on 
previous results in a variety of ways. 

Within these $HH$ searches, I was responsible for the development of the background 
estimation and corresponding uncertainties, one of the most important aspects of the $4b$ channel. For the resonant 
search, I contributed strongly to the development of the selection, the documentation, and final development of the 
result. No significant excesses are observed, but a low mass mis-modeling has been reduced relative to the early 
Run 2 searches, and the limits set by the \bbbb channel are very competitive with the other full Run 2 ATLAS results.

For the non-resonant search, in addition to the background estimation, I was behind many of the changes that were made 
relative to the resonant search. These changes result in a very competitive limit of $4.4(5.9)$ times the Standard Model 
observed (expected), a result which is both competitive with the other leading channels, and beats a projection of 
the early Run 2 result (from luminosity scaling) by 30 (40)~\%. This represents a significant improvement in the analysis 
strategy, and sets the stage for a very promising set of $HH$ results, both from the full Run 2 combination and looking 
towards Run 3. Though an excess is seen for values of $\kappa_{\lambda} \geq 5$, such an excess is demonstrated to be due 
to the low mass regime, where \bbbb has limited sensitivity.

With the contributions of this thesis work, the \bbbb channel remains one of the leading $HH$ channels in sensitivity. With 
further increases in luminosity, as well as development of even more advanced analysis techniques 
(see, e.g., Appendix \ref{app:future}), the future is bright for $HH$ at the LHC. 