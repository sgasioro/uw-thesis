\chapter{The Anatomy of an LHC Search}
In this thesis so far, we have set the theoretical foundation for the work 
carried out at the LHC. We have described how one may translate between this 
theoretical foundation and what we are actually able to observe with the ATLAS 
detector. We have further stepped through the process of simulating production of 
specific physics processes and their appearance in our detector, allowing us to 
describe how a hypothetical physics model would be seen in our experiment. The question 
then becomes: all of these pieces are on the table, what do we do with them? This chapter
attempts to answer exactly that, setting up a roadmap for assembling these pieces into 
a statement about the universe.

\section{Object Selection and Identification}
As described in Chapter \ref{chap:reconstruction}, there is a complicated set of steps for 
going from electrical signals in a detector to physics objects.

\section{Defining a Signal Region}
\section{Background Estimation}
\section{Uncertainty Estimation}
\section{Hypothesis Testing}
