\chapter{The Standard Model of Particle Physics}
\label{chap:intro-SM}
\emph{``I may be bad, but I'm perfectly good at it'' - Rihanna re: the Standard Model (SM), or so I've been told}\\\\

The Standard Model of Particle Physics (SM) is a monumental historical achievement, providing a 
formalism with which one may describe everything from the physics of everyday experience to the 
physics that is studied at very high energies at the Large Hadron Collider (Chapter \ref{chap:experiment}). 
In this chapter, we will provide a brief overview of the pieces that go into the 
construction of such a model. The primary focus of this thesis is searches for pair production of 
Higgs bosons decaying to four $b$-quarks. Consequently, we will pay particular attention to the 
relevant pieces of the Higgs Mechanism, as well as the theory behind searches at a hadronic collider.

\section{Particles and Fields}
What is a particle? The Standard Model describes a set of fundamental, point-like, objects (shown in Figure SM FIGURE). 
These objects have distinguishing characteristics (e.g., mass and spin). These objects interact in very specific 
ways. The set of objects and their interactions result in a set of observable effects, and these effects are the basis of a field of experimental physics. 

The effects of these objects and their interactions are familiar as fundamental forces: electromagnetism (photons, 
electrons), the strong interaction (quarks, gluons), the weak interaction (neutrinos, $W$ and $Z$ bosons). Gravity is not 
described in this model, as the weakest, with effects most relevant on much larger distance scales than the rest. However, 
the description of these other three is powerful -- verifying and searching for cracks in this description is a large 
effort, and the topic of this thesis.

The formalism for describing these particles and their interactions is that of quantum field theory. Classical field theory is most familiar in the context of, e.g., electromagnetism -- an electric field exists in some region of space, and a charged point-particle experiences a force characterized by the charge of the point-particle and the magnitude of the field at the location of the point-particle in spacetime. The same language translates to quantum field theory. Here, each particle is represented by a quantum field describing its influence on a region of spacetime. Particles also have charges which describe the forces they experience when interacting with other particles (other quantum fields). Most familiar is electric charge -- however this applies to e.g., the strong interaction as well, where particles have an associated \emph{color charge} describing behavior under the strong force.

Particles are observed to behave in different ways under these different forces. These behaviors respect certain \emph{symmetries}, which are most naturally described in the language of group theory. The respective fields, charges, and 
generators of these symmetry groups are the basic pieces of the SM Lagrangian, which describes the full dynamics of the theory. In the following, we will build up the basic components of this Lagrangian.

\subsection{Quantum Electrodynamics}

\subsection{The Weak Force}

\subsection{Quantum Chromodynamics}

\subsection{The Higgs mechanism}






