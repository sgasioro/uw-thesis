\chapter{The Standard Model of Particle Physics}
\label{chap:intro-SM}
\emph{``I may be bad, but I'm perfectly good at it'' - Rihanna re: the Standard Model (SM), or so I've been told}\\\\

The Standard Model of Particle Physics (SM) is a monumental historical achievement, providing a 
formalism with which one may describe everything from the physics of everyday experience to the 
physics that is studied at very high energies at the Large Hadron Collider (Chapter \ref{chap:experiment}). 
In this chapter, we will provide a brief overview of the pieces that go into the 
construction of such a model. The primary focus of this thesis is searches for pair production of 
Higgs bosons decaying to four $b$-quarks. Consequently, we will pay particular attention to the 
relevant pieces of the Higgs Mechanism, as well as the theory behind searches at a hadronic collider.

\section{Particles and Fields}
What is a particle? The Standard Model describes a set of fundamental, point-like, objects (shown in Figure SM FIGURE). 
These objects have distinguishing characteristics (e.g., mass and spin). These objects interact in very specific 
ways. The set of objects and their interactions result in a set of observable effects, and these effects are the basis of a field of experimental physics. 

The effects of these objects and their interactions are familiar as fundamental forces: electromagnetism (photons, 
electrons), the strong interaction (quarks, gluons), the weak interaction (neutrinos, $W$ and $Z$ bosons). Gravity is not 
described in this model, as the weakest, with effects most relevant on much larger distance scales than the rest. However, 
the description of these other three is powerful -- verifying and searching for cracks in this description is a large 
effort, and the topic of this thesis.

The formalism for describing these particles and their interactions is that of quantum field theory. Classical field theory is most familiar in the context of, e.g., electromagnetism -- an electric field exists in some region of space, and a charged point-particle experiences a force characterized by the charge of the point-particle and the magnitude of the field at the location of the point-particle in spacetime. The same language translates to quantum field theory. Here, particles are described in terms of quantum fields in some region of spacetime. These fields have associated charges which describe the forces they experience when interacting with other quantum fields. Most familiar is electric charge -- however this applies to e.g., the strong interaction as well, where quantum fields have an associated \emph{color charge} describing behavior under the strong force.

Particles are observed to behave in different ways under different forces. These behaviors respect certain \emph{symmetries}, which are most naturally described in the language of group theory. The respective fields, charges, and 
generators of these symmetry groups are the basic pieces of the SM Lagrangian, which describes the full dynamics of the theory. In the following, we will build up the basic components of this Lagrangian.

\subsection{Quantum Electrodynamics}
Classical electrodynamics is familiar to the general physics audience: electric ($\vec{E}$) and magnetic ($\vec{B}$) 
fields are used to describe behavior of particles with charge $q$ moving with velocity $\vec{v}$, with 
forces described as $\vec{F} =q\vec{E} + q \vec{v} \times \vec{B}$. Hints at some more fundamental properties of
electric and magnetic fields come via a simple thought experiment: in a frame of reference moving along with 
the particle at velocity $\vec{v}$, the particle would appear to be standing still, and therefore have no 
magnetic force exerted. Therefore a \emph{relativistic} formulation of the theory is required. This is most 
easily accomplished with a repackaging: the fundamental objects are no longer classical fields but the electric 
and magnetic \emph{potentials}: $\phi$ and $\vec{A}$ respectively, with
\begin{align}
\vec{E} &= -\grad\phi + \frac{\partial \vec{A}}{\partial t}\\
\vec{B} &= \grad \times \vec{A}
\end{align}

It is then natural to fully repackage into a relativistic \emph{four-vector}: $A_{\mu} = (\phi, \vec{A})$. 
Considering $\partial^{\mu} = (\frac{\partial}{\partial t}, \grad)$, the $x$ components of these above 
two equations become:
\todo{Check signs and units!! Add Jackson reference?}
\begin{align}
E_{x} &= -\frac{\partial \phi}{\partial x} + \frac{\partial A_{x}}{\partial t} = ( \partial^0 A^1 - \partial^1 A^0)\\
B_{x} &= \frac{\partial A_{z}}{\partial y} - \frac{\partial A_{y}}{\partial z} = (\partial^2 A^3 - \partial^3 A^2)
\end{align}

This is naturally suggestive of a second rank, antisymmetric tensor to describe both the electric and magnetic 
fields (the \emph{field strength tensor}), defined as:
\begin{equation}
F^{\alpha\beta} = \partial^{\alpha}A^{\beta} - \partial^{\beta} A^{\alpha}
\end{equation}

Defining a four-current as $J_{\mu} = (q, \vec{J})$, with $q$ standard electric charge, $\vec{J}$ standard electric 
current, conservation of charge may be expressed via the continuity equation
\begin{equation}
\partial_{\mu}J^\mu = 0
\end{equation}
and all of classical electromagnetism may be packaged into the Lagrangian density:
\begin{equation}
\mathcal{L} = -\frac{1}{4} F_{\mu\nu}F^{\mu\nu} - J^{\mu}A_{\mu}.
\end{equation}

This gets us partway to our goal, but is entirely classical - the description is of classical fields and 
point charges, not of quantum fields and particles. To reframe this, let us go back to the zoomed out view 
of the particles of the Standard Model. Two of the most familiar objects associated with electromagnetism 
are electrons: spin-1/2 particles with charge $e$, mass $m$, and photons: massless spin-1 particles which are
the "pieces" of electromagnetic radiation.

We know that electrons experience electromagnetic interactions with other objects. Given this, and the 
fact that such interactions must be transmitted \emph{somehow} between e.g. two electrons, it seems natural 
that these interactions are facilitated by electromagnetic radiation. More specifically, we may think of 
photons as \emph{mediators} of the electromagnetic force. It follows, then, that a description of 
electromagnetism on the level of particles must involve a description of both the ``source" particles 
(e.g. electrons), the mediators (photons), and their interactions. Further, this description must be 
consistent with the classically derived dynamics described above.

The description of spin-1/2 particles is courtesy of Paul Dirac \todo{ref}


\subsection{The Weak Force}

\subsection{Quantum Chromodynamics}

\subsection{The Higgs Mechanism}
In the previous sections 







