\chapter{Introduction: The Standard Model of Particle Physics}
\label{chap:intro-SM}
\emph{``I may be bad, but I'm perfectly good at it'' - Rihanna re: the Standard Model (SM), or so I've been told}\\\\

The Standard Model of Particle Physics (SM) is a monumental historical achievement, providing a 
formalism with which one may describe everything from the physics of everyday experience to the 
physics that is studied at very high energies at the Large Hadron Collider (Chapter \ref{chap:experiment}). 
In this chapter, we will provide a brief overview of the pieces that go into the 
construction of such a model, and expound upon some features relevant to the work done for this thesis.

\section{Particles and Fields}
We will begin with a question that is important for a particle physicist to understand: what is a particle? 
The experimental physicist (or at least the author of this thesis) might, upon hearing this question, 
say ``it is something that we can see in our detector'' and point you to Figure SM TABLE as an example 
of the particles that we have observed.

Okay, one might say, what characteristics of these mythical objects are actually observed? This depends on the 
detector (see Chapter \ref{chap:experiment}), but some natural characteristics that come to mind are electrical 
charge -- namely, how the particle moves in an electric or magnetic field -- and mass, or energy, which that particle 
may deposit in some detector system. 

If these are all there is, the definition of particle is then ``something with an electrical charge and a mass.'' 
What, then, of the photon? This is electrically neutral and massless, how is this observed? 
``Well it still interacts with other particles!'' the thesis author would yell! So we have arrived at a definition of a
particle as some abstract object with defined characteristics (such as mass and electrical charge), which interacts 
with other such objects in a well defined way.

This is now a reasonable starting point for a more formal definition, for which we need a bit more math. 

\subsection{Physics is just Linear Algebra}


\section{Path Integrals and Lagrangians}
\section{}




