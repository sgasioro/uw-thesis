\chapter{Background Estimation}
\label{chap:bbbb-background}

After the event selection described above, there are
two major backgrounds, QCD and \ttbar. A very similar approach is used 
for both the resonant and the non-resonant analyses, with some small modifications. 
This approach is notably fully data-driven, which is warranted due to 
the flexibility of the estimation method, as well as the high relative proportion of 
QCD background ($>90~\%$), and allows for the use of machine learning methods in the construction of
the background estimate. However, it sacrifices an explicit treatment of the
\ttbar component. Performance of the background estimate on the \ttbar component
is checked explicitly, and minimal impact due to \ttbar 
mis-modeling is seen.

Contributions of single Higgs processes and $\PZ\PZ$ are found 
to be negligible, and the Standard Model $\higgs\higgs$ background
is found to have no impact on the resonant search.

The foundation of the background estimate lies in the derivation of a
reweighting function which matches the kinematics of events with exactly two
\btagged jets to those of events in the higher tagged regions (events with three or four 
\btagged jets). The reweighting function and overall normalization are derived in 
the control region. Systematic bias of this estimate is assessed in the validation 
region. 

For the resonant analysis, the systematic bias is a bias due to extrapolation:
the validation region lies between the control and signal regions. For the 
non-resonant analysis, the bias instead comes from different possible interpolations
of the signal region kinematics -- given the choice of nominal estimate, the validation
region is a conceptually equivalent, but maximally different, signal region estimate.

\section{The Two Tag Region}

Events in data with exactly two b-tagged jets are used for the data driven 
background estimate. The hypothesis here is that, due to the presence of 
multiple \btagged jets, the kinematics of such events are similar to the 
kinematics of events in higher b-tagged regions (i.e. events with three and 
four \btagged jets, respectively), and any differences can be corrected by a 
reweighting procedure. The region with three \btagged jets is split into 
two $b$-tagging regions, as described in Section \ref{sec:analysis-selection}, 
with the 3$b$ + 1 loose region used as an additional signal 
region. The lower tagged 3$b$ component (3$b$ + 1 fail) is reserved for validation
of the background modelling procedure. Events with fewer than two \btagged jets are not 
used for this analysis, as they are relatively more different from the higher tag regions.

The nominal event selection requires at least four jets in order to form Higgs
candidates. For the four tag region, these are the four highest $p_{T}$ \btagged 
jets. For the three tag regions, these jets are the three \btagged jets, plus the 
highest $p_{T}$ jet satisfying a loosened $b$-tagging requirement. Similarly, and following 
the approach of the resonant analysis, the two tag region uses the two \btagged jets 
and the two highest $p_{T}$ non-tagged jets to form Higgs candidates. Combinatoric bias 
from selection of different numbers of \btagged jets is corrected as a part of the kinematic 
reweighting procedure through the reweighting of the total number of jets in the event. In this way, 
the full event selection may be run on two tagged events. 

\section{Kinematic Reweighting}
The set of two tagged data events is the fundamental piece of the data driven
background estimate. However, kinematic differences from the four tag region
exist and must be corrected in order for this estimate to be useful. Binned 
approaches based on ratios of histograms have been previously considered~\cite{EXOT-2016-31},~\cite{HDBS-2018-18}, 
but are limited in their handling of correlations 
between variables and by the ``curse of dimensionality'', i.e. the dataset
becomes sparser and sparser in ``reweighting space'' as the number of dimensions
in which to reweight increases, limiting the number of variables used for reweighting. 
This leads either to an unstable fit result (overfitting with finely grained bins) or a 
lower quality fit result (underfitting with coarse bins).

Note that even some machine learning methods such as Boosted Decision Trees (BDTs)~\cite{BDT-RW}, may suffer
from this curse of dimensionality, as the depth of each decision tree used is limited by 
the available statistics after each set of corresponding selections (cf. binning in a 
more sophisticated way), limiting the expressivity of the learned reweighting function.

To solve these issues, a neural network based reweighting procedure is used
here. This is a truly multivariate approach, allowing for proper treatment of
variable correlations. It further overcomes the issues associated with binned
approaches by learning the reweighting function directly, allowing for greater
sensitivity to local differences and helping to avoid the curse of
dimensionality.


\subsection{Neural Network Reweighting}
Let $p_{4b}(x)$ and $p_{2b}(x)$ be the probability density functions for four and two tag data respectively across 
some input variables $x$. The problem of learning the reweighting function between two and four tag data 
is then the problem of learning a function $w(x)$ such that
\begin{equation}
p_{2b}(x) \cdot w(x) = p_{4b}(x)
\end{equation}
from which it follows that
\begin{equation}
w(x) = \frac{p_{4b}(x)}{p_{2b}(x)}.
\end{equation}

This falls into the domain of density ratio estimation, for which there are a variety
of approaches. The method considered here is modified from ~\cite{NNloss, NNloss1}, and depends on
a loss function of the form
\begin{equation}
\mathcal{L}(R(x)) = \mathbb{E}_{x\sim p_{2b}}[\sqrt{R(x)}]
+\mathbb{E}_{x\sim p_{4b}}[\frac{1}{\sqrt{R(x)}}].
\end{equation}
where $R(x)$ is some estimator dependent on $x$ and $\mathbb{E}_{x\sim p_{2b}}$ and 
$\mathbb{E}_{x\sim p_{4b}}$ are the expectation values with respect to the 2b and 4b probability 
densities. A neural network trained with such a loss function has the objective of finding
the estimator, $R(x)$, that minimizes this loss. It is straightforward to show that
\begin{equation}
\arg \min_{R}\mathcal{L}(R(x)) = \frac{p_{4b}(x)}{p_{2b}(x)}
\end{equation}
which is exactly the form of the desired reweighting function.

In practice, to avoid imposing explicit positivity constraints, the substitution
$Q(x) \equiv \log R(x)$ is made. The loss function then takes the equivalent form
\begin{equation}
\mathcal{L}(Q(x)) = \mathbb{E}_{x\sim p_{2b}}[\sqrt{e^{Q(x)}}]
+\mathbb{E}_{x\sim p_{4b}}[\frac{1}{\sqrt{e^{Q(x)}}}],
\end{equation}
with solution
\begin{equation}
\arg \min_{Q}\mathcal{L}(Q(x)) = \log\frac{p_{4b}(x)}{p_{2b}(x)}.
\end{equation}
Taking the exponent then results in the desired reweighting function.

Note that similar methods for density ratio estimation are available~\cite{NNRW-Nachman},
e.g. from a more standard binary cross-entropy loss. However, these were found to
perform no better than the formulation presented here.

\subsection{Variables and Results}
The neural network is trained on a variety of variables sensitive to two vs.
four tag differences. To help bring out these differences, the natural logarithm 
of some of the variables with a large, local change is taken. The 
set of training variables used for the resonant analysis is
\begin{enumerate}
	\item $\log(p_T)$ of the 4th leading Higgs candidate jet
	\item $\log(p_T)$ of the 2nd leading Higgs candidate jet
	\item $\log(\Delta R)$ between the closest two Higgs candidate jets
	\item $\log(\Delta R)$ between the other two Higgs candidate jets
	\item Average absolute value of $\eta$ across the four Higgs candidate jets
	\item $\log(p_T)$ of the di-Higgs system.
	\item $\Delta R$ between the two Higgs candidates
	\item $\Delta \phi$ between the jets in the leading Higgs candidate
	\item $\Delta \phi$ between the jets in the subleading Higgs candidate
	\item $\log(X_{\PW\Pqt})$, where $X_{\PW\Pqt}$ is the variable used for the top veto
	\item Number of jets in the event.
\end{enumerate}
The non-resonant analysis uses an identical set of variables with two notable changes
\begin{enumerate}
	\item The definition of $X_{\PW\Pqt}$ differs from the resonant definition (as described 
	in Section \ref{sec:kinematic-cuts}).
	\item An integer encoding of the two trigger categories is used as an input (variable which 
	takes on the value 0 or 1 corresponding to each of the two categories). This was found to improve
	a mis-modeling near the tradeoff in \mhh of the two buckets.
\end{enumerate}

The neural network used for both resonant and non-resonant reweighting has three 
densely connected hidden layers of 50 nodes each with ReLU activation functions and a single node 
linear output. This configuration demonstrates good performance in the modelling 
of a variety of relevant variables, including \mhh, when compared to a 
range of networks of similar size.

In practice, a given training of the reweighting neural network is subject to variation
due to training statistics and initial conditions. An uncertainty is assigned to account
for this (Chapter \ref{chap:bbbb-uncert}), which relies on training an ensemble of
reweighting networks~\cite{DeepEnsembles}. To increase the stability of the background estimate,
the median of the predicted weight for each event is calculated across the ensemble.
This median is then used as the nominal background estimate. This approach is indeed 
seen to be much more stable and to demonstrate a better overall performance than a 
single arbitrary training. Each ensemble used for this analysis consists of 100 
neural networks, trained as described in Chapter \ref{chap:bbbb-uncert}.

The training of the ensemble used for the nominal estimate is done in the kinematic
Control Region. The prediction of these networks in the Signal Region is then used
for the nominal background estimate. In addition, a separate ensemble of networks is 
trained in the Validation Region. The difference between the prediction of the nominal 
estimate and the estimate from the VR derived networks in the Signal Region is used to 
assign a systematic uncertainty. Further details on this systematic uncertainty are discussed 
in Chapter \ref{chap:bbbb-uncert}.
Note that although the same procedure is used for both Control and Validation Region trained 
networks, only the median estimate from the VR derived reweighting is used for assessing a 
systematic -- no additional ``uncertainty on the uncertainty'' from 
VR ensemble variation is applied.

Each reweighted estimate is normalized such that the reweighted $2b$ yield matches the $4b$ 
yield in the corresponding training region. Note that this applies to each of the networks 
used in each ensemble, where the normalization factor is also subject to the procedure described 
in Chapter \ref{chap:bbbb-uncert}. As the median over these normalized weights is not guaranteed 
to preserve this normalization, a further correction is applied such that the $2b$ yield, after 
the median weights are applied, matches the $4b$ yield in the corresponding training region. As no 
pre-processing is applied to correct for the class imbalance between $2b$ and $4b$ events entering 
the training, this ratio of number of $4b$ events ($n(4b)$) over number of $2b$ events ($n(2b)$) is 
folded into the learned weights. Correspondingly, the set of normalization factors described above is 
near $1$ and the learned weights are centered around $n(4b) / n(2b)$ (roughly 0.01 over the full dataset). 
This normalization procedure applies for all instances of the reweighting (e.g. those used for validations 
in Section \ref{sec:bkgd-validation}), with appropriate substitutions of reweighting origin 
(here $2b$) and reweighting target (here $4b$).

Note that, due to different trigger and pileup selections during each year, 
the reweighting is trained on each year separately. An approach of training all of the years 
together with a one-hot encoding was explored, but was found to have minimal 
benefit over the split years approach, and in fact to increase 
the systematic bias of the corresponding background estimate.
Because of this, and because trigger selections for each year significantly impact the kinematics of each year, 
such that categorizing by year is expected to reflect groupings of kinematically similar events and 
to provide a meaningful degree of freedom in the signal extraction fit, the split-year approach is
kept.

The control region closure for the 2018 dataset is shown for a large variety of variables for 
the resonant search in Figures \ref{fig:res-Control18-1} 
through \ref{fig:res-Control18-8} and for the non-resonant search in Figures \ref{fig:nonres-Control184b-1} 
through \ref{fig:nonres-Control184b-8} for $4b$ and Figures \ref{fig:nonres-Control183b1l-1} 
through \ref{fig:nonres-Control183b1l-8} for $3b1l$. The impact of this control 
region derived reweighting in the validation region for the same set of variables is included in 
Appendix \ref{app:reweight-plots}. Demonstration of the performance of the reweighting on the 
final discriminant distributions is included in Chapter \ref{chap:bbbb-results}.
2018 is chosen because it is the largest subset of the data on which the year-by-year reweighting 
is trained. The other years are omitted here for brevity, but demonstrate very similar results. Generally 
good performance is seen, with some occasional mis-modeling. For the resonant 
search, this is most notable in the case of individual jet $p_{T}$. Such mis-modeling may be corrected 
by including the variables in the input set, but this was found to not improve the modeling of $m_{HH}$, 
and so is not done here. This mis-modeling is notable for the non-resonant search in the leading Higgs candidate 
jet $p_{T}$, and is a direct consequence of the trigger category input, which improves modeling of $m_{HH}$.
Results are similar for other years, but are not included here for brevity. 

One other salient feature of the non-resonant plots is the distributions of $m_{H1}$ and $m_{H2}$, which 
emphasize the quadrant region definitions -- the control region has a peak around \SI{125}{\GeV} in $m_{H1}$, 
which may be thought of as ``signal region-like'', motivating this alignment, though consequently the 
distribution of $m_{H2}$ is quite bimodal. The reverse is true for the validation region.

\def\secforfig{background}
\def\figsversion{ggF}
\def\NNTtag{thesis}
\def\weightlabel{bstrap-med}
\def\norwlabel{norw-bstrap-med}
\def\yr{18}

\foreach \region in {Control}{
	\begin{figure}[ht]
	    \centering
	    \subfloat{\label{fig:res-dRjj1-\region\yr-norw}%
	            \includegraphics[width=0.48\textwidth]{figures/\NNTtag-dRjj-1-\norwlabel-\region-NN-\yr.pdf}
	    }   
	    \subfloat{\label{fig:res-dRjj1-\region\yr}%
	            \includegraphics[width=0.48\textwidth]{figures/\NNTtag-dRjj-1-\weightlabel-\region-NN-\yr.pdf}
	    }   

	    \subfloat{\label{fig:res-dRjj2-\region\yr-norw}%
	            \includegraphics[width=0.48\textwidth]{figures/\NNTtag-dRjj-2-\norwlabel-\region-NN-\yr.pdf}
	    }   
	    \subfloat{\label{fig:res-dRjj2-\region\yr}%
	            \includegraphics[width=0.48\textwidth]{figures/\NNTtag-dRjj-2-\weightlabel-\region-NN-\yr.pdf}
	    }   

	    \subfloat[Before Reweighting]{\label{fig:res-etai-\region\yr-norw}%
	            \includegraphics[width=0.48\textwidth]{figures/\NNTtag-eta-i-\norwlabel-\region-NN-\yr.pdf}
	    }   
	    \subfloat[After Reweighting]{\label{fig:res-etai-\region\yr}%
	            \includegraphics[width=0.48\textwidth]{figures/\NNTtag-eta-i-\weightlabel-\region-NN-\yr.pdf}
	    } 
	    \caption{\textbf{Resonant Search}: Distributions of $\Delta R$ between the closest Higgs Candidate
	      jets, $\Delta R$ between the other two, and average absolute value of HC
	      jet $\eta$ before (left) and after (right) CR derived reweighting for the 20\yr\space \region\space Region.
	      \label{fig:res-\region\yr-1}}
	\end{figure}

	\begin{figure}[ht]
	    \centering
	    \subfloat{\label{fig:res-pT4-\region\yr-norw}%
	            \includegraphics[width=0.48\textwidth]{figures/\NNTtag-pT-4-\norwlabel-\region-NN-\yr.pdf}
	    }   
	    \subfloat{\label{fig:res-pT4-\region\yr}%
	            \includegraphics[width=0.48\textwidth]{figures/\NNTtag-pT-4-\weightlabel-\region-NN-\yr.pdf}
	    }   

	    \subfloat{\label{fig:res-pT2-\region\yr-norw}%
	            \includegraphics[width=0.48\textwidth]{figures/\NNTtag-pT-2-\norwlabel-\region-NN-\yr.pdf}
	    }   
	    \subfloat{\label{fig:res-pT2-\region\yr}%
	            \includegraphics[width=0.48\textwidth]{figures/\NNTtag-pT-2-\weightlabel-\region-NN-\yr.pdf}
	    }   

	    \subfloat[Before Reweighting]{\label{fig:res-pthh-\region\yr-norw}%
	            \includegraphics[width=0.48\textwidth]{figures/\NNTtag-pt-hh-\norwlabel-\region-NN-\yr.pdf}
	    }   
	    \subfloat[After Reweighting]{\label{fig:res-pthh-\region\yr}%
	            \includegraphics[width=0.48\textwidth]{figures/\NNTtag-pt-hh-\weightlabel-\region-NN-\yr.pdf}
	    }
	    \caption{\textbf{Resonant Search}: Distributions of $p_{T}$ of the 2nd and 4th leading Higgs Candidate jets and the $p_{T}$ of the di-Higgs system before (left) and after (right) CR derived reweighting for the 20\yr\space \region\space Region.
	    \label{fig:res-\region\yr-2}}
	\end{figure}

	\begin{figure}[ht]
	    \centering
	    \subfloat[Before Reweighting]{\label{fig:res-njets-\region\yr-norw}%
	            \includegraphics[width=0.48\textwidth]{figures/\NNTtag-njets-\norwlabel-\region-NN-\yr.pdf}
	    }
	    \subfloat[After Reweighting]{\label{fig:res-njets-\region\yr}%
	            \includegraphics[width=0.48\textwidth]{figures/\NNTtag-njets-\weightlabel-\region-NN-\yr.pdf}
	    }
	    \caption{\textbf{Resonant Search}: Distributions of the number of jets before (left) and after (right) CR derived reweighting for the 20\yr\space \region\space Region. A minimum of 4 jets is required in each event in order to form Higgs candidates.
	    \label{fig:res-\region\yr-njet}}
	\end{figure}

	\begin{figure}[ht]
	    \centering
	    \subfloat{\label{fig:res-pT1-\region\yr-norw}%
	            \includegraphics[width=0.48\textwidth]{figures/\NNTtag-pT-1-\norwlabel-\region-NN-\yr.pdf}
	    }   
	    \subfloat{\label{fig:res-pT1-\region\yr}%
	            \includegraphics[width=0.48\textwidth]{figures/\NNTtag-pT-1-\weightlabel-\region-NN-\yr.pdf}
	    }   

	    \subfloat{\label{fig:res-pT3-\region\yr-norw}%
	            \includegraphics[width=0.48\textwidth]{figures/\NNTtag-pT-3-\norwlabel-\region-NN-\yr.pdf}
	    }   
	    \subfloat{\label{fig:res-pT3-\region\yr}%
	            \includegraphics[width=0.48\textwidth]{figures/\NNTtag-pT-3-\weightlabel-\region-NN-\yr.pdf}
	    }   

	    \subfloat[Before Reweighting]{\label{fig:res-dRhh-\region\yr-norw}%
	            \includegraphics[width=0.48\textwidth]{figures/\NNTtag-dR-hh-\norwlabel-\region-NN-\yr.pdf}
	    }   
	    \subfloat[After Reweighting]{\label{fig:res-dRhh-\region\yr}%
	            \includegraphics[width=0.48\textwidth]{figures/\NNTtag-dR-hh-\weightlabel-\region-NN-\yr.pdf}
	    }
	    \caption{\textbf{Resonant Search}: Distributions of $p_{T}$ of the 1st and 3rd leading Higgs Candidate jets and $\Delta R$ between Higgs candidates before (left) and after (right) CR derived reweighting for the 20\yr\space \region\space Region.
	    \label{fig:res-\region\yr-3}}
	\end{figure}

	\begin{figure}[ht]
	    \centering
	    \subfloat{\label{fig:res-eta1-\region\yr-norw}%
	            \includegraphics[width=0.48\textwidth]{figures/\NNTtag-eta-1-\norwlabel-\region-NN-\yr.pdf}
	    }   
	    \subfloat{\label{fig:res-eta1-\region\yr}%
	            \includegraphics[width=0.48\textwidth]{figures/\NNTtag-eta-1-\weightlabel-\region-NN-\yr.pdf}
	    }   

	    \subfloat{\label{fig:res-eta2-\region\yr-norw}%
	            \includegraphics[width=0.48\textwidth]{figures/\NNTtag-eta-2-\norwlabel-\region-NN-\yr.pdf}
	    }   
	    \subfloat{\label{fig:res-eta2-\region\yr}%
	            \includegraphics[width=0.48\textwidth]{figures/\NNTtag-eta-2-\weightlabel-\region-NN-\yr.pdf}
	    }   

	    \subfloat[Before Reweighting]{\label{fig:res-eta3-\region\yr-norw}%
	            \includegraphics[width=0.48\textwidth]{figures/\NNTtag-eta-3-\norwlabel-\region-NN-\yr.pdf}
	    }   
	    \subfloat[After Reweighting]{\label{fig:res-eta3-\region\yr}%
	            \includegraphics[width=0.48\textwidth]{figures/\NNTtag-eta-3-\weightlabel-\region-NN-\yr.pdf}
	    }
	    \caption{\textbf{Resonant Search}: Distributions of $\eta$ of the 1st, 2nd, and 3rd leading Higgs Candidate jets before (left) and after (right) CR derived reweighting for the 20\yr\space \region\space Region.
	    \label{fig:res-\region\yr-4}}
	\end{figure}

	\begin{figure}[ht]
	    \centering
	    \subfloat{\label{fig:res-eta4-\region\yr-norw}%
	            \includegraphics[width=0.48\textwidth]{figures/\NNTtag-eta-4-\norwlabel-\region-NN-\yr.pdf}
	    }   
	    \subfloat{\label{fig:res-eta4-\region\yr}%
	            \includegraphics[width=0.48\textwidth]{figures/\NNTtag-eta-4-\weightlabel-\region-NN-\yr.pdf}
	    }   

	    \subfloat{\label{fig:res-pTh1-\region\yr-norw}%
	            \includegraphics[width=0.48\textwidth]{figures/\NNTtag-pT-h1-\norwlabel-\region-NN-\yr.pdf}
	    }   
	    \subfloat{\label{fig:res-pTh1-\region\yr}%
	            \includegraphics[width=0.48\textwidth]{figures/\NNTtag-pT-h1-\weightlabel-\region-NN-\yr.pdf}
	    }   

	    \subfloat[Before Reweighting]{\label{fig:res-pTh2-\region\yr-norw}%
	            \includegraphics[width=0.48\textwidth]{figures/\NNTtag-pT-h2-\norwlabel-\region-NN-\yr.pdf}
	    }   
	    \subfloat[After Reweighting]{\label{fig:res-pTh2-\region\yr}%
	            \includegraphics[width=0.48\textwidth]{figures/\NNTtag-pT-h2-\weightlabel-\region-NN-\yr.pdf}
	    }
	    \caption{\textbf{Resonant Search}: Distributions of $\eta$ of the 4th leading Higgs Candidate jet and the $p_{T}$ of the leading and subleading Higgs candidates before (left) and after (right) CR derived reweighting for the 20\yr\space \region\space Region.
	    \label{fig:res-\region\yr-5}}
	\end{figure}

	\begin{figure}[ht]
	    \centering
	    \subfloat{\label{fig:res-mh1-\region\yr-norw}%
	            \includegraphics[width=0.48\textwidth]{figures/\NNTtag-m-h1-\norwlabel-\region-NN-\yr.pdf}
	    }   
	    \subfloat{\label{fig:res-mh1-\region\yr}%
	            \includegraphics[width=0.48\textwidth]{figures/\NNTtag-m-h1-\weightlabel-\region-NN-\yr.pdf}
	    }   

	    \subfloat{\label{fig:res-mh2-\region\yr-norw}%
	            \includegraphics[width=0.48\textwidth]{figures/\NNTtag-m-h2-\norwlabel-\region-NN-\yr.pdf}
	    }   
	    \subfloat{\label{fig:res-mh2-\region\yr}%
	            \includegraphics[width=0.48\textwidth]{figures/\NNTtag-m-h2-\weightlabel-\region-NN-\yr.pdf}
	    }   

	    \subfloat[Before Reweighting]{\label{fig:res-mhh-\region\yr-norw}%
	            \includegraphics[width=0.48\textwidth]{figures/\NNTtag-m-hh-\norwlabel-\region-NN-\yr.pdf}
	    }   
	    \subfloat[After Reweighting]{\label{fig:res-mhh-\region\yr}%
	            \includegraphics[width=0.48\textwidth]{figures/\NNTtag-m-hh-\weightlabel-\region-NN-\yr.pdf}
	    }
	    \caption{\textbf{Resonant Search}: Distributions of mass of the leading and subleading Higgs candidates and of the di-Higgs system before (left) and after (right) CR derived reweighting for the 20\yr\space \region\space Region.
	    \label{fig:res-\region\yr-6}}
	\end{figure}

	\begin{figure}[ht]
	    \centering
	   
	    \subfloat{\label{fig:res-dPhi-h1-\region\yr-norw}%
	            \includegraphics[width=0.48\textwidth]{figures/\NNTtag-dPhi-h1-\norwlabel-\region-NN-\yr.pdf}
	    }   
	    \subfloat{\label{fig:res-dPhi-h1-\region\yr}%
	            \includegraphics[width=0.48\textwidth]{figures/\NNTtag-dPhi-h1-\weightlabel-\region-NN-\yr.pdf}
	    }   

	    \subfloat[Before Reweighting]{\label{fig:res-dPhi-h2-\region\yr-norw}%
	            \includegraphics[width=0.48\textwidth]{figures/\NNTtag-dPhi-h2-\norwlabel-\region-NN-\yr.pdf}
	    }   
	    \subfloat[After Reweighting]{\label{fig:res-dPhi-h2-\region\yr}%
	            \includegraphics[width=0.48\textwidth]{figures/\NNTtag-dPhi-h2-\weightlabel-\region-NN-\yr.pdf}
	    }
	    \caption{\textbf{Resonant Search}: Distributions of $\Delta \phi$ between jets in the leading and subleading Higgs candidates before (left) and after (right) CR derived reweighting for the 20\yr\space \region\space Region.
	    \label{fig:res-\region\yr-7}}
	\end{figure}

	\begin{figure}[ht]
	    \centering
	    \subfloat[Before Reweighting]{\label{fig:res-Xwt-\region\yr-norw}%
	            \includegraphics[width=0.48\textwidth]{figures/\NNTtag-X-wt-\norwlabel-\region-NN-\yr.pdf}
	    }
	    \subfloat[After Reweighting]{\label{fig:res-Xwt-\region\yr}%
	            \includegraphics[width=0.48\textwidth]{figures/\NNTtag-X-wt-\weightlabel-\region-NN-\yr.pdf}
	    }
	    \caption{\textbf{Resonant Search}: Distributions of the top veto variable, $X_{\PW\Pqt}$, before (left) and after (right) CR derived reweighting for the 20\yr\space \region\space Region. Reweighting is done after the cut on this variable is applied
	    \label{fig:res-\region\yr-8}}
	\end{figure}
}

\input{chapters/nonres-bkg-template-CR}