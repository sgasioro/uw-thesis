\chapter{Di-Higgs Phenomenology and Physics Beyond the Standard Model}
This thesis focuses on searches for di-Higgs production in the \bbbb final state. In this chapter,
we will provide a brief overview of the practical theoretical information motivating such searches. 
Though the searches test for physics beyond the Standard Model, particularly in the search for 
resonances, the goal of the experimental results is to be somewhat agnostic to particular 
theoretical frameworks. An in depth treatment of such models is therefore beyond the scope 
of this thesis, though we will attempt to provide a grounding for the models that we consider.

\section{Intro to Di-Higgs}
Di-Higgs searches can be split into two major theoretical categories: \emph{resonant searches}, in which a physical resonance is produced that subsequently decays into two Higgs bosons, and a \emph{non-resonant searches} in which no physical resonance is produced, but where the $HH$ production cross section has a contribution from an exchange of a \emph{virtual} or \emph{off-shell} particle. 

The focus of this thesis is gluon initiated processes -- in the case of di-Higgs this is termed gluon-glon fusion
(ggF). $\higgs\higgs$ production may also occur via vector boson fusion \todo{cite https://atlas.web.cern.ch/Atlas/GROUPS/PHYSICS/PAPERS/HDBS-2018-18/}. However the cross section for such production is significantly smaller. Representative Feynman diagrams are shown in\todo{add diagrams}.

\todo{show heavier couples more strongly to Higgs - just fill in Yukawa above}. However, the top quark has a mass of \SI{173}{\GeV}, whereas the $\higgs$ has a mass of \SI{125}{\GeV}, such that \HepProcess{\higgs \to \Pqt\Paqt} is kinematically 
disfavored. \HepProcess{\higgs \to \Pqb\Paqb} is therefore the dominant fermionic Higgs decay mode, and, in fact, 
the dominant overall decay mode, with a branching fraction of around 58~\%. The dominant top quark Yukawa coupling to 
the \higgs does play a role in \higgs production, however -- gluon-gluon fusion is dominated by processes including 
a top loop.

The single \higgs properties translate to $\higgs\higgs$ production, with \HepProcess{\higgs\higgs \to \bbbb} accounting
for around 34~\% of all $\higgs\higgs$ decays. The \higgs\higgs branching fractions are shown in Figure \todo{add}.

\section{Resonant $HH$ Searches}
Resonant di-Higgs production is predicted in a variety of extensions to the
Standard Model. In particular, due to \todo{constraints}, this thesis presents
searches for both spin-0 and spin-2 resonances. Each are implemented in a particular 
theoretical context, but set up experimental results for generic searches.

The spin-2 signal considered is implemented within the bulk Randall-Sundrum (RS)
model~\cite{Gravitons}, which features spin-2 Kaluza-Klein gravitons,
\PGrav, that are produced via gluon-fusion and which may decay to a pair of Higgs bosons.
This model was considered in the early Run 2 analysis \todo{cite}, and was excluded 
across much of the relevant mass range. 

The primary theoretical focus of this work is therefore the spin-0 result, which 
is implemented as a generic resonance with width below detector resolution. Scalar 
resonances are interesting, for instance, in the context of two Higgs doublet models~\cite{2HDM}, which 
posit the existence of a second Higgs doublet. This leads to the existence of five scalar
particles in the Higgs sector -- roughly, two complex doublets provide eight degrees of freedom, three of 
which are ``eaten'' by the electroweak bosons, leaving five degrees of freedom which may correspond
to physical fields.

\todo{check out other theory motivation - there are some good HH talks}

\section{Non-resonant $HH$ Searches}
Non-resonant $HH$ production is predicted by the Standard Model via the trilinear coupling discussed above,
as well as via production in a fermion loop. More explicitly, after electroweak symmetry breaking, we have \todo{add Yukawa}
\begin{align}
\mathcal{L}_{SM} &\supset -\lambda v^2h^2 - \lambda v h^3 - \frac{1}{4}\lambda h^4\\
&= -\frac{1}{2}m_{H}^2 - \lambda_{HHH}^{SM}vh^3 - \lambda_{HHHH}^{SM}h^4
\end{align}
where $m_{H} = \sqrt{2\lambda v^2}$ so that 
\begin{equation}
\lambda_{HHH}^{SM} = \frac{m_{H}^2}{2v^2}.
\end{equation}
The mass of the SM Higgs boson has been experimentally measured to be \SI{125}{\GeV} \todo{cite}, and 
the vacuum expectation value $v=$\SI{246}{\GeV} has a precise determination from the muon lifetime \todo{cite}. 
This coupling is therefore precisely predicted in the Standard Model, such that an observed deviation from 
this prediction would be a clear sign of new physics. 

The relevant diagrams for non-resonant $HH$ production are shown in Figure \todo{add diagrams}.
Notably, the diagrams \emph{interfere} with each other, which can be easily seen by counting the fermion lines.
\todo{Some good stuff in here https://arxiv.org/pdf/1504.05596.pdf}.

For the searches presented here, the quark couplings to the Higgs are considered to be consistent with the Standard 
Model value, with measurements of the dominant top Yukawa coupling left to more sensitive direct measurements \todo{cite 
e.g. https://arxiv.org/pdf/2009.07123.pdf}. Variations of the trilinear coupling away from the Standard 
Model are considered, however. Such variations are parametrized via 
\begin{equation}
\kappa_{\lambda} = \frac{\lambda_{HHH}}{\lambda_{HHH}^{SM}}
\end{equation}
where $\lambda_{HHH}$ is a varied coupling, whereas $\lambda_{HHH}^{SM}$ is the Standard Model prediction, given by
As this variation only impacts the \emph{triangle} diagram, significant and interesting effects are observed due to this interference. Examples of the impact of this tradeoff on the di-Higgs invariant mass are shown in \todo{include plot}. Generally speaking, for positive values of $\kappa_{\lambda}$, more events are predicted at low mass, whereas for negative values of $\kappa_{\lambda}$, more events are predicted at high mass.
